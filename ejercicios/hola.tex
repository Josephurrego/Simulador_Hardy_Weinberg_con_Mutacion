\documentclass{article}
\usepackage[utf8]{inputenc}
\usepackage[spanish]{babel}
\usepackage{amsmath}
\usepackage{amssymb}
\usepackage{enumitem}
\usepackage{booktabs}
\usepackage{multirow}
\usepackage{hyperref}
\hypersetup{
	colorlinks=true,
	linkcolor=blue,
	filecolor=magenta,      
	urlcolor=cyan,
}

\title{Solución de los ejercicios sobre el Equilibrio de Hardy-Weinberg}
\author{}
\date{}

\begin{document}
	
	\maketitle
	
	\section*{Página 10 (Ejercicios 1-5)}
	
	\subsection*{Ejercicio 1}
	\begin{enumerate}[label=\alph*)]
		\item \( \Pr(E') = 1 - \Pr(E) \)
		
		\textbf{Procedimiento}: Por la propiedad de probabilidad total, la suma de las probabilidades de un evento y su complemento es 1:
		\[
		\Pr(E) + \Pr(E') = 1 \implies \Pr(E') = 1 - \Pr(E)
		\]
		
		\item \( \Pr(F) = \frac{2}{3} \)
		
		\textbf{Procedimiento}: Dado un dado justo, el evento \( E \) (números divisibles por 3) es \( E = \{3,6\} \), entonces:
		\[
		\Pr(E) = \frac{2}{6} = \frac{1}{3}
		\]
		\( F \) es el complemento de \( E \) (no divisible por 3):
		\[
		\Pr(F) = 1 - \Pr(E) = 1 - \frac{1}{3} = \frac{2}{3}
		\]
	\end{enumerate}
	
	\subsection*{Ejercicio 2}
	\begin{enumerate}[label=\alph*)]
		\item \( \Pr(\text{todas niñas}) = \left( \frac{1}{2} \right)^4 = \frac{1}{16} = 0.0625 \)
		
		\textbf{Procedimiento}: Asumiendo independencia y probabilidad uniforme de género:
		\[
		\Pr(\text{niña}) = \frac{1}{2} \implies \Pr(\text{4 niñas}) = \left( \frac{1}{2} \right)^4
		\]
		
		\item \( \Pr(\text{al menos un niño}) = 1 - \frac{1}{16} = \frac{15}{16} = 0.9375 \)
		
		\textbf{Procedimiento}: El evento complementario de "al menos un niño" es "todas niñas":
		\[
		\Pr(\text{al menos un niño}) = 1 - \Pr(\text{todas niñas})
		\]
	\end{enumerate}
	
	\subsection*{Ejercicio 3}
	\textbf{Procedimiento general}: Para una razón hombres:mujeres \( m:f \):
	\[
	p(\text{hombre}) = \frac{m}{m+f}, \quad p(\text{mujer}) = \frac{f}{m+f}
	\]
	
	\begin{enumerate}[label=\alph*)]
		\item \textbf{EE.UU.} (1052 H : 1000 M):
		\[
		p(\text{mujer}) = \frac{1000}{2052} \approx 0.4873 \implies \Pr(\text{4 niñas}) = (0.4873)^4 \approx 0.0565
		\]
		
		\item \textbf{Grecia} (1073 H : 1000 M):
		\[
		p(\text{mujer}) = \frac{1000}{2073} \approx 0.4824 \implies \Pr(\text{4 niñas}) = (0.4824)^4 \approx 0.0542
		\]
		
		\item \textbf{Chile} (1043 H : 1000 M):
		\[
		p(\text{mujer}) = \frac{1000}{2043} \approx 0.4895 \implies \Pr(\text{4 niñas}) = (0.4895)^4 \approx 0.0576
		\]
	\end{enumerate}
	
	\subsection*{Ejercicio 4}
	Dado un dado no estándar con caras: 1R, 2R, 3R, 4N, 5N, 6N.
	\begin{enumerate}[label=\alph*)]
		\item \( \Pr(A) = \Pr(\text{par}) = \frac{3}{6} = 0.5 \), \( \Pr(B) = \Pr(\text{roja}) = \frac{3}{6} = 0.5 \)
		
		\item \( \Pr(A \cap B) = \Pr(\{2\}) = \frac{1}{6} \approx 0.1667 \) (único resultado par y rojo)
		
		\item Independencia: \( \Pr(A) \cdot \Pr(B) = 0.25 \neq 0.1667 \implies \) \textbf{no independientes}.
	\end{enumerate}
	
	\subsection*{Ejercicio 5}
	Espacio muestral: 36 resultados equiprobables.
	\begin{enumerate}[label=\alph*)]
		\item Probabilidades marginales:
		\[
		\Pr(A) = \frac{18}{36} = 0.5, \quad \Pr(B) = \frac{6}{36} = \frac{1}{6}, \quad \Pr(C) = \frac{5}{36} \approx 0.1389
		\]
		
		\item Probabilidades conjuntas:
		\[
		\Pr(A \cap B) = \frac{3}{36} = \frac{1}{12}, \quad \Pr(A \cap C) = \frac{3}{36} = \frac{1}{12}
		\]
		
		\item Independencia \( A \) y \( B \): \( 0.5 \times \frac{1}{6} = \frac{1}{12} = \Pr(A \cap B) \implies \) \textbf{independientes}.
		
		\item Independencia \( A \) y \( C \): \( 0.5 \times \frac{5}{36} \approx 0.0694 \neq 0.0833 \implies \) \textbf{no independientes}.
	\end{enumerate}
	
	\section*{Ejercicio 6}
	Padres: \( (Rr, tt) \) y \( (rr, Tt) \). Gametos equiprobables:
	\begin{itemize}
		\item Padre 1: \( Rt \), \( rt \) (prob. 0.5 cada uno)
		\item Padre 2: \( rT \), \( rt \) (prob. 0.5 cada uno)
	\end{itemize}
	
	Cuadro de Punnett:
	\begin{center}
		\begin{tabular}{c|c|c}
			& \( rT \) (0.5) & \( rt \) (0.5) \\
			\hline
			\( Rt \) (0.5) & \( RrTt \) & \( Rrtt \) \\
			\hline
			\( rt \) (0.5) & \( rrTt \) & \( rrtt \) \\
		\end{tabular}
	\end{center}
	
	\begin{enumerate}[label=\alph*)]
		\item Cuadro como arriba.
		\item \( \Pr(\text{roja}) = \Pr(R\_) = \frac{2}{4} = 0.5 \)
		\item \( \Pr(\text{corta}) = \Pr(tt) = \frac{2}{4} = 0.5 \)
		\item \( \Pr(\text{roja y corta}) = \Pr(Rrtt) = \frac{1}{4} = 0.25 \)
		\item Independencia: \( 0.5 \times 0.5 = 0.25 = \Pr(\text{roja y corta}) \implies \) \textbf{independientes}.
		\item \( \Pr(\text{roja o corta}) = 0.5 + 0.5 - 0.25 = 0.75 \)
		\item No, porque el evento complementario incluiría otros genotipos como \( RrTt \).
	\end{enumerate}
	

	\section*{Problema 7}
	Supongamos que las plantas cruzadas son ambas de tipo \( (Rr, Tt) \).
	
	\subsection*{a) Cuadro de Punnett}
	Cada parental produce gametos: \( RT \), \( Rt \), \( rT \), \( rt \) (cada uno con probabilidad \( \frac{1}{4} \)).
	
	\vspace{0.5cm}
	\begin{tabular}{c|cccc}
		\toprule
		& \multicolumn{4}{c}{Gametos Parental 2} \\
		\cmidrule(lr){2-5}
		Gametos Parental 1 & \( RT \) & \( Rt \) & \( rT \) & \( rt \) \\
		\midrule
		\( RT \) & RRTT & RRTt & RrTT & RrTt \\
		\( Rt \) & RRTt & RRtt & RrTt & \textbf{Rrtt} \\
		\( rT \) & RrTT & RrTt & rrTT & rrTt \\
		\( rt \) & RrTt & \textbf{Rrtt} & rrTt & \textbf{rrtt} \\
		\bottomrule
	\end{tabular}
	
	\subsection*{b) Probabilidad de flores rojas}
	Las flores rojas corresponden a genotipos \( R\_ \) (al menos un alelo dominante \( R \)). En el cuadro hay 12 combinaciones con flores rojas de 16 posibles:
	
	\[
	P(\text{rojo}) = \frac{12}{16} = \frac{3}{4} = 0.75
	\]
	
	\subsection*{c) Probabilidad de tallos cortos}
	Los tallos cortos corresponden a genotipos \( tt \) (homocigoto recesivo). En el cuadro hay 4 combinaciones con tallos cortos:
	
	\[
	P(\text{corto}) = \frac{4}{16} = \frac{1}{4} = 0.25
	\]
	
	\subsection*{d) Probabilidad de tallos cortos y flores rojas}
	Corresponde a genotipos \( R\_ tt \). En el cuadro hay 3 combinaciones que cumplen ambas condiciones:
	
	\[
	P(\text{rojo y corto}) = \frac{3}{16} = 0.1875
	\]
	
	\textbf{Nota:} La probabilidad también puede calcularse por multiplicación de eventos independientes:
	\[
	P(\text{rojo}) \times P(\text{corto}) = \frac{3}{4} \times \frac{1}{4} = \frac{3}{16}
	\]
	\subsection*{e) ¿flores rojas y tallos cortos son eventos independientes?}
	Dos eventos son independientes si la probabilidad de que ocurran juntos es igual al producto de sus probabilidades individuales.
	
	Calculamos:
	\[
	P(\text{rojo}) \times P(\text{corto}) = \frac{3}{4} \times \frac{1}{4} = \frac{3}{16}
	\]
	Comparando con la probabilidad conjunta:
	\[
	P(\text{rojo y corto}) = \frac{3}{16}
	\]
	Como \(\frac{3}{16} = \frac{3}{16}\), concluimos que son eventos independientes.
	
	\textbf{Respuesta:} Sí, son eventos independientes.
	
	

	
	\section*{Punto 8. Cruce de plantas (Rr, yy, Tt) y (rr, Yy, Tt)}
	
	\subsection*{a) Separación de alelos (gametos posibles)}
	
	\textbf{Planta 1 (Rr, yy, Tt):}
	\begin{itemize}
		\item Forma de semilla: R o r
		\item Color de semilla: y (única posibilidad)
		\item Altura: T o t
	\end{itemize}
	Gametos posibles: $(R, y, T)$, $(R, y, t)$, $(r, y, T)$, $(r, y, t)$
	
	\textbf{Planta 2 (rr, Yy, Tt):}
	\begin{itemize}
		\item Forma de semilla: r (única posibilidad)
		\item Color de semilla: Y o y
		\item Altura: T o t
	\end{itemize}
	Gametos posibles: $(r, Y, T)$, $(r, Y, t)$, $(r, y, T)$, $(r, y, t)$
	
	\subsection*{b) Cuadro de Punnett}
	
	\begin{center}
		\renewcommand{\arraystretch}{1.2}
		\begin{tabular}{c|c|c|c|c|}
			\cline{2-5}
			& \multicolumn{4}{c|}{\textbf{Gametos Planta 2}} \\
			\cline{2-5}
			& $(r, Y, T)$ & $(r, Y, t)$ & $(r, y, T)$ & $(r, y, t)$ \\
			\hline
			\multicolumn{1}{|c|}{$(R, y, T)$} & $(Rr, Yy, TT)$ & $(Rr, Yy, Tt)$ & $(Rr, yy, TT)$ & $(Rr, yy, Tt)$ \\
			\hline
			\multicolumn{1}{|c|}{$(R, y, t)$} & $(Rr, Yy, Tt)$ & $(Rr, Yy, tt)$ & $(Rr, yy, Tt)$ & $(Rr, yy, tt)$ \\
			\hline
			\multicolumn{1}{|c|}{$(r, y, T)$} & $(rr, Yy, TT)$ & $(rr, Yy, Tt)$ & $(rr, yy, TT)$ & $(rr, yy, Tt)$ \\
			\hline
			\multicolumn{1}{|c|}{$(r, y, t)$} & $(rr, Yy, Tt)$ & $(rr, Yy, tt)$ & $(rr, yy, Tt)$ & $(rr, yy, tt)$ \\
			\hline
		\end{tabular}
	\end{center}
	Total de descendientes: 16
	
	\subsection*{c) Independencia entre semillas arrugadas y semillas verdes}
	
	Definimos eventos:
	\begin{itemize}
		\item $A$: Semillas arrugadas ($rr$)
		\item $B$: Semillas verdes ($yy$)
	\end{itemize}
	
	Cálculos:
	\begin{align*}
		P(A) &= \frac{8}{16} = \frac{1}{2} \\
		P(B) &= \frac{8}{16} = \frac{1}{2} \\
		P(A \cap B) &= \frac{4}{16} = \frac{1}{4} \\
		P(A) \times P(B) &= \frac{1}{2} \times \frac{1}{2} = \frac{1}{4}
	\end{align*}
	Como $P(A \cap B) = P(A) \times P(B)$, \textbf{los eventos son independientes}.
	
	\subsection*{d) Independencia entre semillas arrugadas y tallos cortos}
	
	Definimos eventos:
	\begin{itemize}
		\item $A$: Semillas arrugadas ($rr$)
		\item $C$: Tallos cortos ($tt$)
	\end{itemize}
	
	Cálculos:
	\begin{align*}
		P(A) &= \frac{8}{16} = \frac{1}{2} \\
		P(C) &= \frac{4}{16} = \frac{1}{4} \\
		P(A \cap C) &= \frac{2}{16} = \frac{1}{8} \\
		P(A) \times P(C) &= \frac{1}{2} \times \frac{1}{4} = \frac{1}{8}
	\end{align*}
	Como $P(A \cap C) = P(A) \times P(C)$, \textbf{los eventos son independientes}.
	
	\subsection*{e) Independencia entre semillas redondas y tallos altos}
	
	Definimos eventos:
	\begin{itemize}
		\item $D$: Semillas redondas ($R\_$)
		\item $E$: Tallos altos ($T\_$)
	\end{itemize}
	
	Cálculos:
	\begin{align*}
		P(D) &= \frac{8}{16} = \frac{1}{2} \\
		P(E) &= \frac{12}{16} = \frac{3}{4} \\
		P(D \cap E) &= \frac{6}{16} = \frac{3}{8} \\
		P(D) \times P(E) &= \frac{1}{2} \times \frac{3}{4} = \frac{3}{8}
	\end{align*}
	Como $P(D \cap E) = P(D) \times P(E)$, \textbf{los eventos son independientes}.
	
	
	\section*{Ejercicio 9}
	Datos: \( MM = 119 \), \( MN = 76 \), \( NN = 13 \), \( n = 208 \).
	
	\begin{enumerate}[label=\alph*)]
		\item Frecuencias genotípicas:
		\[
		\Pr(MM) = \frac{119}{208} \approx 0.5721, \quad \Pr(MN) \approx 0.3654, \quad \Pr(NN) \approx 0.0625
		\]
		
		\item Frecuencias alélicas:
		\[
		\Pr(M) = \frac{2 \times 119 + 76}{416} \approx 0.7548, \quad \Pr(N) \approx 0.2452
		\]
	\end{enumerate}
	

	
	\section*{Ejercicio 10}
	En el ejemplo mencionado, donde las frecuencias alélicas permanecen constantes entre generaciones, \textbf{las frecuencias genotípicas también permanecen constantes} una vez alcanzado el equilibrio de Hardy-Weinberg. Esto se debe a los principios fundamentales del equilibrio:
	
	\subsection*{Explicación:}
	\begin{enumerate}
		\item \textbf{Equilibrio de Hardy-Weinberg}:
		\begin{itemize}
			\item Cuando una población cumple con las condiciones de Hardy-Weinberg (tamaño poblacional grande, apareamiento aleatorio, sin mutaciones, migración o selección natural), las frecuencias genotípicas se estabilizan en una generación y permanecen constantes en generaciones posteriores.
			\item Las frecuencias genotípicas siguen la distribución binomial:
			\begin{align*}
				p^2 & \quad \text{(homocigotos dominantes, AA)}, \\
				2pq & \quad \text{(heterocigotos, Aa)}, \\
				q^2 & \quad \text{(homocigotos recesivos, aa)},
			\end{align*}
			donde \( p \) y \( q \) son las frecuencias alélicas (\( p + q = 1 \)).
		\end{itemize}
		
		\item \textbf{Comparación entre generaciones}:
		\begin{itemize}
			\item \textbf{Tabla 4} (Generación parental): 
			\begin{itemize}
				\item Muestra frecuencias genotípicas iniciales (que pueden no estar en equilibrio).
			\end{itemize}
			\item \textbf{Tabla 7} (Generación filial):
			\begin{itemize}
				\item Si las frecuencias alélicas no cambiaron (\( p \) y \( q \) constantes), y el apareamiento es aleatorio, las frecuencias genotípicas en la generación filial habrán alcanzado \( p^2 \), \( 2pq \), y \( q^2 \), manteniéndose estables en todas las generaciones futuras.
			\end{itemize}
		\end{itemize}
	\end{enumerate}
	

	\section*{Ejercicio 11}
	Albinismo: \( \Pr(aa) = q^2 = 0.000016 \).
	
	\textbf{Procedimiento}:
	\[
	q = \sqrt{0.000016} = 0.004, \quad p = 1 - q = 0.996
	\]
	\[
	\Pr(AA) = p^2 = (0.996)^2 \approx 0.992016, \quad \Pr(Aa) = 2pq \approx 0.007968
	\]
	
	\section*{Ejercicio 12}
	Verificar equilibrio HW:
	\begin{enumerate}[label=\alph*)]
		\item \( p = 0.4 \), \( q = 0.6 \), \( p^2 = 0.16 \), \( 2pq = 0.48 \), \( q^2 = 0.36 \implies \textbf{sí} \)
		\item \( p = 0.625 \), \( p^2 = 0.390625 \neq 0.50 \implies \textbf{no} \)
		\item \( p = 0.5 \), \( p^2 = 0.25 \neq 0 \implies \textbf{no} \)
		\item \( q = 1 \), \( q^2 = 1 \implies \textbf{sí} \)
	\end{enumerate}
	
	\section*{Ejercicio 13}
	No enrollar lengua: \( \Pr(rr) = q^2 = (0.6)^2 = 0.36 \). Enrollar: \( 1 - 0.36 = 0.64 \).
	
	\section*{Ejercicio 14}
	Si \( \Pr(AA) = \Pr(aa) = x \) y \( \Pr(Aa) = 1 - 2x \), entonces \( p^2 = q^2 \implies p = q = 0.5 \). Luego \( x = (0.5)^2 = 0.25 \).
	
	\section*{Ejercicio 15}
	Cálculo de \( p = \Pr(M) \), \( q = \Pr(N) \) y verificación:
	\begin{center}
		\begin{tabular}{l|c|c|c}
			Población & \( p \) & \( q \) & ¿Equilibrio? \\
			\hline
			Esquimal & 0.913 & 0.087 & Sí (cercano) \\
			Aborigen Australiano & 0.176 & 0.824 & Aproximadamente \\
			Egipcio & 0.5225 & 0.4775 & Sí (cercano) \\
			Alemán & 0.5505 & 0.4495 & No \\
			Chino & 0.575 & 0.425 & Sí \\
			Nigeriano & 0.5485 & 0.4515 & Sí \\
		\end{tabular}
	\end{center}
	
	\section*{Página 24 (Ejercicios 16-17)}
	
	\subsection*{Ejercicio 16}
	Guisantes amarillos (recesivo \( pp \)): \( q = \sqrt{\Pr(\text{amarillo})} \)
	\begin{itemize}
		\item Guyancourt: \( q \approx 0.6928 \), \( \Pr(\text{heterocigoto}) = 2pq \approx 0.4256 \)
		\item Lonchez: \( q \approx 0.5840 \), \( 2pq \approx 0.4858 \)
		\item Peyresourde: \( q \approx 0.9148 \), \( 2pq \approx 0.1558 \)
	\end{itemize}
	
	\subsection*{Ejercicio 17}
	Fibrosis quística: \( \Pr(cc) = q^2 = 0.0004 \)
	\[
	q = 0.02, \quad p = 0.98, \quad \Pr(CC) = p^2 = 0.9604, \quad \Pr(Cc) = 2pq = 0.0392
	\]
	\textbf{¿Apareamiento aleatorio?} Sí, portadores asintomáticos y afectados raros.
	
	\section*{Página 26 (Ejercicios 18-19)}
	
	\subsection*{Ejercicio 18 (Navajo)}
	Sistema ABO: \( \Pr(O) = r^2 = 0.775 \), \( \Pr(B) = 0 \)
	\[
	r = \sqrt{0.775} \approx 0.8803, \quad q = 0, \quad p = 1 - r \approx 0.1197
	\]
	
	\subsection*{Ejercicio 19 (Esquimal)}
	\( \Pr(O) = r^2 = 0.411 \), \( \Pr(B) = q^2 + 2qr = 0.035 \)
	\begin{enumerate}[label=\alph*)]
		\item \( r \approx 0.6412 \), resolver \( p^2 + 2pr = 0.538 \implies p \approx 0.33295 \), \( q \approx 0.02585 \)
		\item \( r \approx 0.6412 \), resolver \( q^2 + 2qr = 0.035 \implies q \approx 0.0267 \), \( p \approx 0.3321 \)
		\item \( \Pr(AB) = 2pq \approx 0.0172 \) vs. observado 0.014 (discrepancia por muestreo).
	\end{enumerate}
	
	\section*{Página 29 (Ejercicio 20)}
	
	\subsection*{Ejercicio 20 (Rh)}
	\begin{enumerate}[label=\alph*)]
		\item General: \( q = \sqrt{0.145} \approx 0.3808 \), \( \Pr(RR) = p^2 \approx 0.3834 \), \( \Pr(Rr) = 2pq \approx 0.4718 \)
		\item Vascos: \( q = \sqrt{0.43} \approx 0.6557 \), \( \Pr(RR) \approx 0.1185 \), \( \Pr(Rr) \approx 0.4515 \)
	\end{enumerate}
	
	\section*{Página 31 (Ejercicios 21-22)}
	
	\subsection*{Ejercicio 21}
	Homozygoto recesivo estéril: \( q_n = \frac{q_0}{1 + n q_0} \), \( q_0 = 0.05 \)
	\begin{enumerate}[label=\alph*)]
		\item \( q_{10} = \frac{0.05}{1 + 10 \times 0.05} = \frac{0.05}{1.5} \approx 0.0333 \)
		\item \( \frac{q_0}{1 + n q_0} = \frac{q_0}{2} \implies n = \frac{1}{q_0} = 20 \) generaciones.
	\end{enumerate}
	
	\subsection*{Ejercicio 22}
	\begin{enumerate}[label=\alph*)]
		\item \( n = \frac{1}{q_0} \) generaciones para reducir \( q \) a la mitad.
		\item Fibrosis quística: \( q_0 = 0.02 \), \( n = 50 \) generaciones, tiempo \( = 50 \times 30 = 1500 \) años.
	\end{enumerate}
	
	\section*{Página 32 (Ejercicio 23)}
	
	\subsection*{Ejercicio 23}
	Homozygotos recesivos no se reproducen. Genotipos iniciales: \( \Pr(AA) = 0.25 \), \( \Pr(Aa) = 0.5 \), \( \Pr(aa) = 0.25 \).
	
	\textbf{Procedimiento}: Solo \( AA \) y \( Aa \) se reproducen. Frecuencia alélica en reproductores:
	\[
	p' = \frac{2 \times 0.25 + 0.5}{2 \times (0.25 + 0.5)} = \frac{1}{1.5} = \frac{2}{3}, \quad q' = \frac{1}{3}
	\]
	Siguiente generación:
	\[
	\Pr(AA) = (p')^2 = \frac{4}{9}, \quad \Pr(Aa) = 2p'q' = \frac{4}{9}, \quad \Pr(aa) = (q')^2 = \frac{1}{9}
	\]
	Equilibrio alcanzado en \( \Pr(AA) = 0.25 \), \( \Pr(Aa) = 0.5 \), \( \Pr(aa) = 0.25 \) con \( p = q = 0.5 \).
	
	\section*{Página 34 (Ejercicio 24)}
	
	\subsection*{Ejercicio 24 (Hibridación)}
	Proporción inicial: 40\% caballos (H), 60\% cebras (Z). Híbridos (X) no se reproducen.
	\begin{enumerate}[label=\alph*)]
		\item Primera generación:
		\[
		\Pr(H) = (0.4)^2 = 0.16, \quad \Pr(X) = 2 \times 0.4 \times 0.6 = 0.48, \quad \Pr(Z) = (0.6)^2 = 0.36
		\]
		
		\item Reproductores (solo H y Z):
		\[
		\Pr(H) = \frac{0.16}{0.52} \approx 0.3077, \quad \Pr(Z) = \frac{0.36}{0.52} \approx 0.6923
		\]
		
		\item Segunda generación:
		\[
		\Pr(H) = (0.3077)^2 \approx 0.0947, \quad \Pr(X) = 2 \times 0.3077 \times 0.6923 \approx 0.4260, \quad \Pr(Z) = (0.6923)^2 \approx 0.4793
		\]
		
		\item Tendencia: Caballos disminuyen, cebras aumentan, híbridos persisten.
	\end{enumerate}
	
\end{document}
